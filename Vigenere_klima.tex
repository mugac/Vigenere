\documentclass{article}

\usepackage[utf8]{inputenc}
\usepackage{amsmath}
\usepackage{amsfonts}
\usepackage{amssymb}

\title{Prolomení Vigenerovy šifry}
\author{Martin Klíma}

\begin{document}

\maketitle

\section{Teorie}
K prolomení vigeneroy šifry je prvně zapotřebí získat délku klíče, který byl použit pro zašifrování. K tomu se dá využít friedmanuv test. Při znalosti délky klíče se zašifrovaný text rozdělí do skupin podle délky klíče (např. pro klíč délky 3 se rozdělí text do tří skupin). Poté je zapořebí pro každou skupinu zjistit posun vůči původní abecedě (pomocí frekvenční analýzy) a tím určit písmeno klíče. 


\end{document}