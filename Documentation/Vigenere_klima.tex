\documentclass{article}

\usepackage[utf8]{inputenc}
\usepackage{amsmath}
\usepackage{amsfonts}
\usepackage{amssymb}

\title{Prolomení Vigenerovy šifry}
\author{Martin Klíma}

\begin{document}

\maketitle

\section{Rychlo teorie}
K prolomení vigeneroy šifry je prvně zapotřebí získat délku klíče, 
který byl použit pro zašifrování. K tomu se dá využít friedmanuv test. 
Při znalosti délky klíče se zašifrovaný text rozdělí do skupin podle 
délky klíče (např. pro klíč délky 3 se rozdělí text do tří skupin). 
Poté je zapořebí pro každou skupinu zjistit posun vůči původní abecedě 
(pomocí frekvenční analýzy) a tím určit posun písmena klíče. Při znalosti 
klíče stačí pouze text pouze dešifrovat.

\section{Teorie}
\subsection{Friedmanův test / Index koincidence}
Pro zjištění délky hesla lze využít Friedmanův test. Tento test spočívá 
v porovnávání indexu koincidence skopin vytovřených ze zašifrovaného 
textu s indexem koincidence jazyka, kterým je psaný otevřený text. Index 
koincidence je pravděpodobnost, že dvě náhodně vybraná písmena z textu 
budou stejná. Pro opravdu náhodný text je index koincidence 1/26. Každý 
jazyk má svůj index koincidence, pro česštinu se odává hodnota 0.058. 
\\\\
Pro výpočet indexu koincidence skupiny písmen délky $n$ se používá vzorec: 
$IK = \sum \limits_{i=1}^{26} \frac{n_i (n_i - 1)}{n (n - 1)}$\\
Abychom udělali průměr pro všechny skupiny, použijeme vzorec:
$IK_{prum} = \frac{1}{n} \sum \limits_{i=1}^{n} IK_i$\\
\end{document}