\documentclass{article}

\usepackage[utf8]{inputenc}
\usepackage{amsmath}
\usepackage{amsfonts}
\usepackage{amssymb}

\title{Prolomení Vigenerovy šifry}
\author{Martin Klíma}

\begin{document}

\maketitle

\section{Rychlo teorie}
K prolomení vigeneroy šifry je prvně zapotřebí získat délku klíče, 
který byl použit pro zašifrování. K tomu se dá využít friedmanuv test. 
Při znalosti délky klíče se zašifrovaný text rozdělí do skupin podle 
délky klíče (např. pro klíč délky 3 se rozdělí text do tří skupin). 
Poté je zapořebí pro každou skupinu zjistit posun vůči původní abecedě 
(pomocí frekvenční analýzy) a tím určit posun písmena klíče. Při znalosti 
klíče stačí pouze text pouze dešifrovat.

\section{Teorie}
\subsection{Friedmanův test / Index koincidence}
Pro zjištění délky hesla lze využít Friedmanův test. Tento test spočívá 
v porovnávání indexu koincidence skopin vytovřených ze zašifrovaného 
textu s indexem koincidence jazyka, kterým je psaný otevřený text. Index 
koincidence je pravděpodobnost, že dvě náhodně vybraná písmena z textu 
budou stejná. Pro opravdu náhodný text je index koincidence 1/26. Každý 
jazyk má svůj index koincidence, pro česštinu se odává hodnota 0.058. 
\\\\
Pro výpočet indexu koincidence skupiny písmen délky $n$ se používá vzorec: 
$IK = \sum \limits_{i=1}^{26} \frac{n_i (n_i - 1)}{n (n - 1)}$\\
Abychom udělali průměr pro všechny skupiny, použijeme vzorec:
$IK_{prum} = \frac{1}{n} \sum \limits_{i=1}^{n} IK_i$\\
\\
Index koincidence můžeme počítat od minimální odhadovaé délky až po 
maximální odhadovanou délku klíče. Také je možné přidat hodnoty možných 
délek získaných například pomocí Kasiského testu.\\
\\
Hodnoty koincidence pro různé delky hesla následně porovnáme s indexem
koincidence pro daný jazyk. Délka klíče je pravděpodobně ta hodnota, která 
se nejvíce shoduje.

\subsection{Prolomení}
Při znalosti délky klíče máme k prolomení šifry více možných postupů
\subsubsection{Útok hrubou silou}
Útok hrubou islou je praktický pouze u krátkých klíčů, jelikož jeho časová 
a výpočetní náročnost exponenciálně roste s délkou klíče.
\subsubsection{Frekvenční analýza}
Způsob pomocí frekvenční analýzy využívá toho, že v každém jazyce se každé 
písmeno vyskytuje ve své vlastní frekvenci oproti čistě náhodnému textu, kde
by frekvence písmen měla být velmi podobná.
\\\\

\begin{tabular}{|c c| c c|}
    \hline
    \multicolumn{4}{|c|}{Frekvence v ČJ} \\
    \hline
    a: & 9.5893 \% & n: & 5.9172 \% \\
    b: & 1.7761 \% & o: & 8.0300 \% \\
    c: & 2.9991 \% & p: & 3.1150 \% \\
    d: & 3.7748 \% & q: & 0.0059 \% \\
    e: & 10.904 \% & r: & 4.3968 \% \\
    f: & 0.1751 \% & s: & 5.5860 \% \\
    g: & 0.2198 \% & t: & 5.3853 \% \\
    h: & 2.4975 \% & u: & 3.5790 \% \\
    i: & 6.6861 \% & v: & 3.9525 \% \\
    j: & 2.3058 \% & w: & 0.0543 \% \\
    k: & 3.5281 \% & x: & 0.0359 \% \\
    l: & 5.7208 \% & y: & 2.8575 \% \\
    m: & 3.6055 \% & z: & 3.3026 \% \\
    \hline
\end{tabular}



\end{document}